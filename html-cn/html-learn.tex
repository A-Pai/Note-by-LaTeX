\documentclass[a4paper,12pt]{report}
% Created by XeLaTeX. TeXLive 2016.
% 宏包
\usepackage[slantfont,boldfont]{xeCJK}
    \setCJKmainfont[BoldFont=SimHei,ItalicFont=KaiTi]{SimSun}
% tabular
\usepackage{array}
% list
\usepackage{ulem}
\usepackage[inline]{enumitem}
   \setlist[enumerate]{label=\arabic* - ,font=\bfseries,itemsep=0pt}
   \setlist[itemize]{label=$\bullet$,font=\bfseries,leftmargin=\parindent}
   \setlist[description]{font=\bfseries\uline}

\usepackage{xcolor}
    \definecolor{commentcolor}{RGB}{85,139,78}
    \definecolor{numbercolor}{RGB}{166,206,168}
    \definecolor{stringcolor}{RGB}{206,145,108}
    \definecolor{keywordcolor}{RGB}{86,156,214}
    \definecolor{backcolor}{RGB}{220,220,220}

\usepackage{listings}
% Avoid copy line numbers of the listing code
\usepackage{accsupp}
\newcommand{\emptyaccsupp}[1]{\BeginAccSupp{ActualText={}}#1\EndAccSupp{}}

% Style Definition
\lstset{% Basic Style
	language=HTML,basicstyle=\small\ttfamily,
	keywordstyle=\color{keywordcolor},
	commentstyle=\color{commentcolor},
	stringstyle=\color{stringcolor},
	showstringspaces=false}

\lstnewenvironment{myht}
{\lstset{
		% Line Number Style
		numbers=left,
		numberstyle=\tiny\emptyaccsupp,
		stepnumber=1,
		% Frame and Background Color
		frame=single,
		framerule=0pt,
		backgroundcolor=\color{backcolor},
		% Spaces
		belowskip=0pt,
		emptylines=1,
		% Escape
		escapeinside=``}}{}
\newcommand{\pyline}[1]{{\lstinline[language=HTML, basicstyle=\ttfamily]{#1}}}
% Title Style & hyper-ref package
\usepackage{titlesec}
\titleformat{\subsubsection}[hang]{\bfseries\large}{$\blacktriangleright$}{0.5em}{}

% Others
\renewcommand{\contentsname}{目\hspace{1em}录}
% 奇怪的小定义
\newcommand{\httag}[1]{\texttt{<#1>\index{<#1>}}}
\newcommand{\dpar}{\\ \mbox{}}	% 空两行
\newcommand{\qd}[1]{\textbf{#1}}	% 强调
\newcommand{\co}[1]{{\bfseries{#1}}}   % Style of concept
\newcommand{\red}[1]{{\color{red}{\uline{#1}}}}
\newcommand{\charef}[1]{第\ref{#1}章}
\newcommand{\secref}[1]{第\ref{#1}节}
\newcommand{\pref}[1]{第\pageref{#1}页}
\newcommand{\fref}[1]{图\ref{#1}}
\newcommand{\tref}[1]{表\ref{#1}}

% 三个自定义列表
\newenvironment{fead}
    {\begin{description}[font=\bfseries\uline]}
    {\end{description}}
\newenvironment{feae}
    {\begin{enumerate}[font=\bfseries,labelindent=0pt]}
    {\end{enumerate}}
\newenvironment{feai}
    {\begin{itemize}[font=\bfseries]}
    {\end{itemize}}
\newenvironment{inlinee}	% Package optioin: inline
{\begin{enumerate*}[label=(\arabic*), font=\rmfamily, before=\unskip{: },itemjoin={{; }},itemjoin*={{, and: }}]}
	{\end{enumerate*}.}

% Manual Space Control
\usepackage{indentfirst}
    \setlength\parskip{10pt}

% Index
\usepackage{imakeidx}
    \makeindex[title={HTML标签索引}]
    \indexsetup{level=\chapter*,
        othercode=\renewcommand{\indexspace}{\smallskip}}
\usepackage[rule=0.4pt,initsep=10pt plus 5pt minus 3pt,%
    columns=3]{idxlayout}
    \renewcommand{\indexfont}{\small}

% The Cover of this manual
\title{简单粗暴HTML}
\author{Kanglong Wu\\ $<$\href{mailto:wklchris@hotmail.com}{wklchris@hotmail.com}$>$}
%\date{July 22, 2016}

\usepackage[colorlinks,bookmarksopen=true,bookmarksnumbered=true]{hyperref}

% ====================
% ======= Main =======
% ====================



\begin{document}

\maketitle

\addcontentsline{toc}{chapter}{目录}
\tableofcontents

\chapter{简介}
\red{本手册中的HTML如无说明,均指HTML4. HTML5和XHTML等不在此讨论。}

HTML即\co{超文本标记语言(Hyper Text Markup Language)},广泛用于网页。它不同于编程语言,主要的语法形式是\co{标签(tag)},并且往往是成对的标签。格式如:
\begin{myht}
<html>
  <body>
...
  </body>
</html>
\end{myht}

位于一对标签内部(即两个尖括号之间)的内容称为\co{元素}。以上的\httag{html}标签定义了整个HTML文档的范围,而\httag{body}标签定义了HTML文档的主体。

HTML对大小写不敏感,但是在后续的版本中可能会使用强制小写。

\section{标题和段落}
标题通过\httag{h1>--<h6}标签进行创建,而段落使用\httag{p}。比如:
\begin{myht}
<h1>Heading 1</h1>
<p>Paragraph 1</p>
\end{myht}

\section{链接和图像}
链接使用\httag{a}标签进行创建,同时需要指定链接地址,以\texttt{href}属性的方式给出。

图像的标签为\httag{img /},属性为\texttt{src}. 注意,这里并没有给出类似\texttt{/img}这样的结束标签,而是在起始标签中加上了空格和斜杠。
\begin{myht}
<a href="http://www.example.com/">Link</a>
<img src="example.jpg" width="100" height="100" />
\end{myht}

更多的内容请分别参考:\hyperref[sec:a]{链接}、\hyperref[sec:img]{图像}。

\section{换行}
换行使用\httag{br /}标签。这是一种特殊的标签,它的元素为空,也并不是成对的。像\httag{img}标签一样,它需要在末尾的\texttt{'>'}前加上空格和斜杠。
\begin{myht}
Line<br />A new line
\end{myht}

\section{水平线}
水平线使用\httag{hr /}标签。
\begin{myht}
<p>Para 1</p>
<hr />
<p>Para 2<p/>
\end{myht}

\section{注释}
注释只要在\texttt{'<'}后面紧跟一个叹号就可以。
\begin{myht}
<!--我是一个注释啊哈哈哈-->
\end{myht}

\section{文本格式}
文本格式有以下标签定义:

\begin{center}
\begin{tabular}{ll}
\httag{b} &	定义粗体文本。\\
\httag{big} & 定义大号字。\\
\httag{em} & 定义着重文字。\\
\httag{i} &	定义斜体字。 \\
\httag{small} & 定义小号字。 \\
\httag{strong} & 定义加重语气。 \\
\httag{sub} & 定义下标字。 \\
\httag{sup} & 定义上标字。 \\
\httag{ins} & 定义插入字。 \\
\httag{del} & 定义删除字。
\end{tabular}
\end{center}

预格式文本\httag{pre}会保留空格,因此用于显示代码很不错:
\begin{myht}
<pre>
for num in lst:
    datalst.append(num)
</pre>
\end{myht}

对于计算机输出的,还有以下几种标签:
\begin{center}
\begin{tabular}{ll}
\httag{code} & 定义计算机代码。\\
\httag{kbd} & 定义键盘输入码。\\
\httag{samp} & 定义代码样本。\\
\httag{tt} & 定义打字机字体代码。 \\
\httag{var} & 定义变量。
\end{tabular}
\end{center}

\section{缩写}
缩写使用\httag{abbr}或者\httag{acronym},并配合\texttt{title}属性。
\begin{myht}
<abbr title="Alphabet">AB</abbr>
<acronym title="ID EST">i.e.</acronym>
\end{myht}

\chapter{进阶}
\section{样式}
通过\httag{head}标签的定义来引用一个外部的css文件:
\begin{myht}
<head>
<link rel="stylesheet" type="text/css" href="mystyle.css">
</head>
\end{myht}

内部的样式也可以直接定义,通过\httag{style}的方式:
\begin{myht}
<head>
<style type="text/css">
body {background-color:red}
p {margin-left:20pt}
</style>
</head>
\end{myht}

更加临时的一种样式方案是通过属性来更改:
\begin{myht}
<p style="color: red; margin-left: 20px">
This is a paragraph
</p>
\end{myht}

\section{链接}
\label{sec:a}
前面已经介绍过链接,标签是\httag{a},\texttt{href}属性就能够应对大多数情况了。其中,元素的部分可以是文本,也可以是别的东西(比如一个图片)。注意:\red{url应以“/”结尾,否则可能会发送两次同一请求}。
\begin{myht}
<a href="url">Link text</a>
\end{myht}

你可以用\texttt{name}定义一个锚,然后用\texttt{href}加上\#号指向它,甚至在另一份HTML文档指向它:
\begin{myht}
<a href="#Me">Visit Here</a>
<h1><a name="ME">This is ME</a></h1>
另一份文档:
<a href="URL#ME">Go to ME</a>
\end{myht}

你还可以用\texttt{target}指定链接被打开的方式。\_{}blank表示新窗口(或选项卡),\_{}self表示同框架(缺省),\_{}parent表示父框架,\_{}top表示整个窗口。
\begin{myht}
<a href="url" target="_blank">Here</a>
\end{myht}

\section{图像}
\label{sec:img}
关于图像使用就是\texttt{<img src="url" />},以及属性width和height,之前已经介绍过。还可设置一个加载失败时会代替图片显示的文本:
\begin{myht}
<img src="url" alt="Fail loading" />
\end{myht}

align标签可以设置行内图像与同行文本的竖直对齐方式,比如默认的图片下缘与文本同是高\texttt{align="bottom"}。此外还可以使用top和middle. 在\texttt{<p>}标签内使用align属性为left或right的图像,可以使之自动浮动到文本的左侧或者右侧。

背景图片使用\httag{body}标签的\texttt{background}属性,一般常用的是jpg或者gif。写在\texttt{body}的起始标签中:
\begin{myht}
<body background="url">
\end{myht}

\section{表格}
表格由\httag{table}标签定义,每行的内容写在一对\httag{tr}标签中,而每行中的单元格写在一对\httag{td}标签中。表头用\httag{th}定义,你也可以省略。下面给出了一行两列的表格:
\begin{myht}
<table border="1">
<tr>
  <th>Col 1</th>
  <th>Col 2</th>
</tr>
<tr>
  <td>1,1</td>
  <td>1,2</td>
</tr>
</table>
\end{myht}

border属性设置为1表示显示表格边框宽度为1。当然了,表头只是特殊的单元格样式,你可以把它写在其他位置,比如第一列。如果你不想加表头而是想要标题,使用\httag{caption}标签,:
\begin{myht}
<table border="3">
<caption>标题</caption>
<tr>
  <td>1</td>
  <td>&nbsp;</td>
</tr>
</table>
\end{myht}

像上面显示的一样——如果表格中有空单元格,不要在\texttt{<td>}标签间留空,而是加上\verb|&nbsp;|。留空的一种替代方案往往是跨行或跨列,使用rowspan或者colspan属性。跨行右侧的列与跨列下方的行不写出。
\begin{myht}
<table border="1">
<tr>
  <th colspan="2">Heading</th>
</tr>
<tr>
  <td>Col 1</td>
  <td>Col 2</td>
</tr>
</table>
\end{myht}

标签\texttt{<table>}除了border属性还有ceilpadding属性,用于在文字和表格边框之间留出空白;ceilspacing属性,用于控制单元格之间的距离。

如果想给表格加上背景,可以在\texttt{<table>}中添加bgcolor属性设置为单色,或者添加background属性设置一张图片。如果这两个属性用于单个\texttt{<td>},那么就只设置单个单元格的背景。

单元格内的水平对齐方式采用align标签,left/right/center属性分别是居左/居右/居中对齐,以及justify两段对齐\footnote{justify属性并不在所有浏览器中都适用。}。默认值是left. 常用的小数点对齐这样写:
\begin{myht}
<td align="char" char=".">1.00</td>
\end{myht}

单元格竖直方向的对齐由valign标签的top, middle, bottom, baseline四种参数确定。默认值middle. 

最后,\texttt{<table>}标签给出了一种方便地设置框线的方式(但在IE中可能不支持),即frame属性。该属性的取值:
\begin{center}
\begin{tabular}{>{\texttt}ll}
void & 无外侧边框。 \\
above & 上边框。 \\
below & 下边框。 \\
hsides & 上下边框。 \\
vsides & 左右边框。 \\
lhs & 左边框。 \\
rhs & 右边框。 \\
box/border & 四侧边框。 
\end{tabular}
\end{center}

\section{列表}
无序列表\httag{ul}和有序列表\httag{ol}内部都使用\httag{li}标签标记每一项。
\begin{myht}
<ul>
  <li>First</li>
  <li>Second</li>
</ul>
\end{myht}

无序列表的项目符号可以通过type属性更改,disc/square/circle三种值。有序列表的编号方式有数字、大小写字母、大小写罗马数字一共1, a, A, i, I五种type. \red{虽然可以直接更改编号样式,但推荐用CSS取代它。}
\begin{myht}
<ol type="A">
  <li>this</li>
  <li>that</li>
</ol>
\end{myht}

解释列表\httag{dl}类似于\LaTeX 中的description环境,使用\httag{dt}标记要解释的内容,\httag{dd}标记解释文本。
\begin{myht}
<dl>
  <dt>HTML</dt>
  <dd>Hyper Text Markup Language</dd>
</dl>
\end{myht}

\chapter{块与布局}
HTML中有两种元素:\co{块元素(block level element)}和\co{内联元素(inline element)}. 区别在于,块元素在使用时会自动在其前方断行,而内联元素则不会。这有点像\LaTeX 中环境和命令的区别。

HTML中有一个没有特殊含义的块元素\httag{div},和一个没有特殊含义的内联元素\httag{span}. 它们在布局设置时很有用。

% 
% ===== Index =====
%

\printindex

\end{document}