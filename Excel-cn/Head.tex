\documentclass[a4paper, zihao=-4, linespread=1]{ctexrep}
\renewcommand{\CTEXthechapter}{\thechapter}

% 页面修正
\usepackage{geometry}
    \geometry{vmargin = 1in}
% 最小行间间距设定
\setlength{\lineskiplimit}{3pt}
\setlength{\lineskip}{3pt}
% 其他宏包
\usepackage[normalem]{ulem}
\usepackage{xcolor}

% 中文支持
% \usepackage[slantfont,boldfont]{xeCJK}
%	\setCJKmainfont[BoldFont=SimHei,ItalicFont=KaiTi]{SimSun}
%	\setCJKmathfont{STXinwei}
% \usepackage{indentfirst}
% 新字体声明
% \setCJKmathfont{KaiTi}
% \newCJKfontfamily[zhxinwei]\xinwei{STXinwei}


% 小定义及重命名
\newcommand{\qd}[1]{\textbf{\uline{#1}}}
\newcommand{\dpar}{\\ \mbox{}}	% 空两行
\newcommand{\RED}[1]{{\color{cyan}{#1}}}

\renewcommand{\contentsname}{目录}
	\renewcommand{\tablename}{表}
	\renewcommand\arraystretch{1.2}	% 表格行距
	\renewcommand{\figurename}{图}


% 编号列表宏包
\usepackage[inline]{enumitem}
	\setlist[enumerate]{font=\bfseries,itemsep=0pt,parsep=0pt,topsep=0pt,partopsep=0pt}
	\setlist[itemize]{font=\bfseries,leftmargin=\parindent,itemsep=0pt,parsep=0pt,topsep=0pt,partopsep=0pt}
	\setlist[description]{font=\bfseries\uline,labelindent=\parindent,itemsep=0pt,parsep=0pt,topsep=0pt,partopsep=0pt}
% 行内列表
\newenvironment{inlinee}
{\begin{enumerate*}[label=(\arabic*), font=\rmfamily, before=\unskip{:},itemjoin={{;}},itemjoin*={{,以及:}}]}
	{\end{enumerate*}。}


% 目录和章节样式
\usepackage{titlesec}
\usepackage{titletoc}   % 用于目录

\titlecontents{chapter}[1.5em]{}{\contentslabel{1.5em}}{\hspace*{-2em}}{\hfill\contentspage}
\titlecontents{section}[3.3em]{}{\contentslabel{1.8em}}
	{\hspace*{-2.3em}}{\titlerule*[8pt]{$\cdot$}\contentspage}
\titlecontents*{subsection}[2.5em]{\small}{\thecontentslabel{} }{}{, \thecontentspage}[;\qquad][.]
% 章节样式
\setcounter{secnumdepth}{3} % 一直到subsubsection
% 扩展章节
\newcommand{\starsec}{\noindent\fbox{\S\textit{注意:本章节是一个扩展阅读章节。}}
	\\ \mbox{}}


% 图表
\setlength{\abovecaptionskip}{5pt}
\setlength{\belowcaptionskip}{3pt}

\usepackage{array,multirow,makecell}
  \setlength\extrarowheight{2pt} % 行高增加
\usepackage{diagbox}
\usepackage{longtable}
\usepackage{graphicx}
  \graphicspath{{./pics/}}
\usepackage{caption}
  % \usepackage{subcaption}
  % \captionsetup[sub]{labelformat=simple}
  % \renewcommand{\thesubtable}{(\alph{subtable})}


% 代码环境
\usepackage{listings}
\usepackage{accsupp}
	\newcommand{\emptyaccsupp}[1]{\BeginAccSupp{ActualText={}}#1\EndAccSupp{}}
\usepackage{tcolorbox}
    \tcbuselibrary{listings,skins,breakable}

% 语法介绍环境
\newenvironment{syntax}{\begin{center}\ttfamily}
    {\end{center}}

% 行内Excel函数
\newcommand{\exstyle}[1]{\texttt{#1}}
\newtcbox{\excel}[1][green]{on line,before upper=\ttfamily,
  arc=0pt,outer arc=0pt,colback=#1!10!white,colframe=#1!50!black,
  boxsep=0pt,left=1pt,right=1pt,top=1pt,bottom=1pt,
  boxrule=0pt,bottomrule=1pt,toprule=1pt}

% Excel Code
\lstdefinelanguage{excellang}{
	classoffset=0, % 蓝色的keyword
	morekeywords={sum, sumif, sumifs, sumproduct, sumsq, sumx2my2, sumx2py2, sumxmy2, 
	              product, mmult, combin, permut, mod, quotient, base, abs, 
	              rand, randbetween, power, sqrt, odd, even, trunc,
	              int, ceiling, floor, round, rounddown, roundup, 
	              subtotal, true, false, and, or, not, if, iferror, 
	              asc, widechar, code, char, concatenate, 
	              proper, lower, upper, value, text, fixed, rept, len,
	              trim, find, replace, search, substitute, left, mid, right,
	              clean, numbervalue, average, averagea, averageif,
	              count, counta, countblank, countif, countifs,
	              max, dmax, min, dmin, small, large, mode, rank, frequency,
	              now, today, date, year, month, day, yearfrac, edate, eomonth, datedif,
	              timevalue, hour, minute, second, weekday, weeknum, workday, networkdays,
	              row, rows, column, columns, indirect, address, offset, transpose, index,
	              formulatext, vlookup, hlookup, match, lookup, choose, hyperlink, iseven, isodd,
	              iserr, iserror, isna, istext, isnumber, cell, isblank, evaluate, 
	              fv, pv, nper, rate, cumipmt, cumprinc, db, ddb, syd, vbd},
	sensitive=false,
	% keywordstyle=\color{blue},
	% classoffset=1,   % 第二关键字(如有)
	% morekeywords={},
	% keywordstyle=\color{envicolor},
	morecomment=[l]{\#},
	morecomment=[s]{/*}{*/},
	morestring=[b]',
	% Escape
	escapeinside=``
}

% 多行Excel代码环境
\newtcblisting{excode}{breakable,skin=bicolor,colback=gray!30!white,
  colbacklower=white,colframe=cyan!75!black,listing only, 
  left=6mm,top=2pt,bottom=2pt,fontupper=\small,
  % listing style
  listing options={language=excellang, keywordstyle=\color{blue}, commentstyle=\ttfamily\color{green!50!black},
  numbers=left,numberstyle=\tiny\color{red!75!black}\emptyaccsupp,
  emptylines=1}
}
\newcommand{\greenmath}[1]{{\color{green!50!black}\begin{math}#1\end{math}}}

% 附录
\renewcommand{\appendixname}{App.}

% 参考文献
\bibliographystyle{plain}
\renewcommand{\bibname}{参考文献}
\usepackage[numindex,numbib]{tocbibind}
\usepackage[square,super,sort&compress]{natbib}

% 引用
\usepackage{hyperref}
    \renewcommand{\figureautorefname}{图}
    \renewcommand{\tableautorefname}{表}
    \renewcommand{\chapterautorefname}{章}
    \renewcommand{\sectionautorefname}{节}
    \renewcommand{\subsectionautorefname}{次节}
\hypersetup{colorlinks, bookmarksopen = true, bookmarksnumbered = true, pdftitle=Excel-cn, pdfauthor=K.L Wu, pdfstartview=FitH}
