\chapter{\mbox{Tikz绘图}*}
其实\LaTeX 本身是有绘图功能的,可以绘制线段、斜率为整数的直线等基础功能。显然这是不能满足绘图需求的。因此在本手册中,对于\LaTeX{}原生的绘图功能不再多做介绍,有兴趣的可以自行了解。本手册主要针对Tikz绘图。\footnote{在\LaTeX 中其实还可以使用PSTricks宏包,是非常强力。XYpic宏包则有时用来画小图。}

PGFPlots/Tikz是两个\LaTeX{}的绘图宏包,功能极其强大。其基本的结构为:
\begin{latex}{}
\documentclass{doc-class}
\usepackage{tikz}
\usetikzlibrary{...}
\begin{document}
  \begin{tikzpicture}
    ...
  \end{tikzpicture}
\end{document}
\end{latex}

废话不多说,进入正题。
\section{简单的实例}
\subsection{直线、网格与点}
\noindent\begin{tabular}{p{0.25\linewidth}l}
\begin{tikzpicture}[baseline=(current bounding box.east)]
  \draw (0,0) -- (1,2);
\end{tikzpicture}
&
\begin{tikzcode}{}
\begin{tikzpicture}
  \draw (0,0) -- (1,2);
\end{tikzpicture}
\end{tikzcode}
\end{tabular}

注意到help lines这个参数使网格绘制更有辅助线的效果,颜色更淡:\vspace{0.5em}

\noindent\begin{tabular}{p{0.25\linewidth}l}
\begin{tikzpicture}[baseline=(current bounding box.east)]
  \draw (0,0) grid (2,1);
  \draw [help lines](0,1) grid (2,3);
\end{tikzpicture}
&
\begin{tikzcode}{}
\begin{tikzpicture}
  \draw (0,0) grid (2,1);
  \draw [help lines](0,1) grid (2,3);
\end{tikzpicture}
\end{tikzcode}
\end{tabular}

点命令也非常好理解。其中\verb+\coordinate+是真正将点的名字和坐标联系起来的命令(相当于``赋值''),而\verb+\node+命令仅是一个标注文本的作用。

注意到第二行\verb+\node+命令,它使得字母B标注在pB点315度的位置。这样的效果往往优于其上一行的\verb+\node+命令的效果。注意:\RED{不要忘记{\texttt{\char92{}node}}命令后面的花括号(即使它是空的)}!

\noindent\begin{tabular}{p{0.25\linewidth}l}
\begin{tikzpicture}[baseline=(current bounding box.east)]
  \coordinate (pA) at (1,0);
  \coordinate (pB) at (2,3);
  \coordinate (pC) at (0,2);
  \node at (pC) {$C$};
  \node[label=315:$B$] at (pB){};
  \draw (pA) -- (pB) -- (pC) -- (pA);
  \draw [help lines](0,0) grid (2,3);
\end{tikzpicture}
&
\begin{tikzcode}{}
\begin{tikzpicture}
  \coordinate (pA) at (1,0);
  \coordinate (pB) at (2,3);
  \coordinate (pC) at (0,2);
  \node at (pC) {$C$};
  \node[label=315:$B$] at (pB){}; 
  \draw (pA) -- (pB) -- (pC) -- (pA);
  \draw [help lines](0,0) grid (2,3);
\end{tikzpicture}
\end{tikzcode}
\end{tabular}

\subsection{拉伸}
拉伸只需要在tikzpicture后添加xscale/yscale/scale的可选参数赋值即可。例如:

\noindent\begin{tabular}{p{0.25\linewidth}l}
\begin{tikzpicture}[baseline=(current bounding box.east),yscale=0.5]
  \coordinate (pA) at (1,0);
  \coordinate (pB) at (2,3);
  \coordinate (pC) at (0,2);
  \draw (pA) -- (pB) -- (pC) -- (pA);
  \draw [help lines](0,0) grid (2,3);
\end{tikzpicture}
&
\begin{tikzcode}{}
\begin{tikzpicture}[yscale=0.5]
  \coordinate (pA) at (1,0);
  \coordinate (pB) at (2,3);
  \coordinate (pC) at (0,2);
  \draw (pA) -- (pB) -- (pC) -- (pA);
  \draw [help lines](0,0) grid (2,3);
\end{tikzpicture}
\end{tikzcode}
\end{tabular}

\subsection{线宽}
线宽可以在tikzpicture后使用``[line width=5pt]''之类的参数进行\RED{全局调整},也可以在每个draw指令下分别地进行调整:

\noindent\begin{tabular}{p{0.25\linewidth}l}
\begin{tikzpicture}[baseline=(current bounding box.east)]
  \coordinate (pA) at (1,0);
  \coordinate (pB) at (2,3);
  \coordinate (pC) at (0,2);
  \node[label=270:$A$] at (pA){};
  \node[label=0:$B$] at (pB){};
  \node[label=180:$C$] at (pC){};
  \draw[ultra thick] (pA) -- (pB);
  \draw[thick] (pB)-- (pC);
  \draw[thin] (pC)-- (pA);
  \draw[ultra thin] (pB) -- (0,0);
  \draw[line width=0.3cm] (pC) -- (2,1);
  \draw [help lines](0,0) grid (2,3);
\end{tikzpicture}
&
\begin{tikzcode}{}
\begin{tikzpicture}
  \coordinate (pA) at (1,0);
  \coordinate (pB) at (2,3);
  \coordinate (pC) at (0,2);
  \node[label=270:$A$] at (pA){};
  \node[label=0:$B$] at (pB){};
  \node[label=180:$C$] at (pC){};
  \draw[ultra thick] (pA) -- (pB);
  \draw[thick] (pB)-- (pC);
  \draw[thin] (pC)-- (pA);
  \draw[ultra thin] (pB) -- (0,0);
  \draw[line width=0.3cm] (pC) -- (2,1);
  \draw [help lines](0,0) grid (2,3);
\end{tikzpicture}
\end{tikzcode}
\end{tabular}

\subsection{填色}
填色的逻辑非常简单,是指向绘制的\qd{闭合}图形内部填色。本例中,draw命令首尾相接地连接三个点,构成一个闭合图形——三角形,因此可以填色:

\noindent\begin{tabular}{p{0.25\linewidth}l}
\begin{tikzpicture}[baseline=(current bounding box.east)]
  \coordinate (pA) at (1,0);
  \coordinate (pB) at (2,3);
  \coordinate (pC) at (0,2);
  \draw[fill=green] (pA) -- (pB) -- (pC) -- (pA);
  \draw[help lines](0,0) grid (2,3);
\end{tikzpicture}
&
\begin{tikzcode}{}
\begin{tikzpicture}
  \coordinate (pA) at (1,0);
  \coordinate (pB) at (2,3);
  \coordinate (pC) at (0,2);
  \draw[fill=green] (pA) -- (pB) -- (pC) -- (pA);
  \draw[help lines] (0,0) grid (2,3);
\end{tikzpicture}
\end{tikzcode}
\end{tabular}

\subsection{点样式}
填色之后图形好看了许多,但是还不够。点标在图中似乎也不看清位置,不过这是有方法解决的:

\noindent\begin{tabular}{p{0.25\linewidth}l}
\begin{tikzpicture}[baseline=(current bounding box.east)]
  \coordinate (pA) at (1,0);
  \coordinate (pB) at (2,3);
  \coordinate (pC) at (0,2);
  \node[label=270:$A$,circle,draw,fill,inner sep=3pt] at (pA){};
  \node[label=0:$B$,circle,draw,fill,inner sep=3pt] at (pB){};
  \node[label=180:$C$,circle,draw,fill,inner sep=3pt] at (pC){};
  \draw[fill=green] (pA) -- (pB) -- (pC) -- (pA);
  \draw[help lines] (0,0) grid (2,3);
\end{tikzpicture}
&
\begin{tikzcode}{}
\begin{tikzpicture}
  \coordinate (pA) at (1,0);
  \coordinate (pB) at (2,3);
  \coordinate (pC) at (0,2);
  \node[label=270:$A$,circle,draw,fill,inner sep=3pt] at (pA){};
  \node[label=0:$B$,circle,draw,fill,inner sep=3pt] at (pB){};
  \node[label=180:$C$,circle,draw,fill,inner sep=3pt] at (pC){};
  \draw[fill=green] (pA) -- (pB) -- (pC) -- (pA);
  \draw[help lines] (0,0) grid (2,3);
\end{tikzpicture}
\end{tikzcode}
\end{tabular}

但是这一长串简直是太复杂了。大概能看懂circle是将点画成圆形,fill表示填充,inner sep表示点的大小。而Tikz给出的\verb+\tikzstyle+能够简化步骤。

还有一点就是,绿色的填充似乎位于点的上方?还是乖乖地调整语句顺序吧。

\noindent\begin{tabular}{p{0.25\linewidth}l}
\begin{tikzpicture}[baseline=(current bounding box.east)]
  \draw[help lines] (0,0) grid (2,3);
  \coordinate (pA) at (1,0);
  \coordinate (pB) at (2,3);
  \coordinate (pC) at (0,2);
  \draw[fill=green] (pA) -- (pB) -- (pC) -- (pA);
  \tikzstyle{every node} = [circle,draw,fill=blue,inner sep=2pt];
  \node[label=270:$A$] at (pA){};
  \node[label=0:$B$] at (pB){};
  \node[label=180:$C$] at (pC){};   
\end{tikzpicture}
&
\begin{tikzcode}{}
\begin{tikzpicture}
  \draw[help lines] (0,0) grid (2,3);
  \coordinate (pA) at (1,0);
  \coordinate (pB) at (2,3);
  \coordinate (pC) at (0,2);
  \draw[fill=green] (pA) -- (pB) -- (pC) -- (pA);
  \tikzstyle{every node} = [circle,draw,fill=blue,inner sep=2pt];
  \node[label=270:$A$] at (pA){};
  \node[label=0:$B$] at (pB){};
  \node[label=180:$C$] at (pC){}; 
\end{tikzpicture}
\end{tikzcode}
\end{tabular}

注意:\verb+\tikzstyle{every node}+控制的是其下方代码的所有node的点样式,不影响其上方代码中的点。

\subsection{线型和颜色}
除了点的定义,线当然也可以定义。线型有dashed/dotted两种,颜色除了默认的也可以通过叹号的形式控制。

\noindent\begin{tabular}{p{0.25\linewidth}l}
\begin{tikzpicture}[baseline=(current bounding box.east)]
  \draw[help lines] (0,0) grid (2,3);
  \coordinate (pA) at (1,0);
  \coordinate (pB) at (2,3);
  \coordinate (pC) at (0,2);
  \draw[dashed, ultra thick] (pA) -- (pB);
  \draw[dotted, red, thick] (pB) -- (pC);
  \draw[blue!30!yellow, ultra thick] (pC) -- (pA);
\end{tikzpicture}
&
\begin{tikzcode}{}
\begin{tikzpicture}
  \draw[help lines] (0,0) grid (2,3);
  \coordinate (pA) at (1,0);
  \coordinate (pB) at (2,3);
  \coordinate (pC) at (0,2);
  \draw[dashed, ultra thick] (pA) -- (pB);
  \draw[dotted, red, thick] (pB) -- (pC);
  \draw[blue!30!yellow, ultra thick] (pC) -- (pA);
\end{tikzpicture}
\end{tikzcode}
\end{tabular}

主要的颜色包括\footnote{以下色块用\texttt{\char92{}tikz\{\char92{}draw[{\it color},line width=9] (0,0) -- (0.5,0);\}绘制得到。}}:
\tikzline{red}\tikzline{green}\tikzline{yellow}\tikzline{blue}\tikzline{cyan}
\tikzline{magenta}\tikzline{black}\tikzline{gray}\tikzline{darkgray}\tikzline{lightgray}
\tikzline{brown}\tikzline{lime}\tikzline{olive}\tikzline{orange}\tikzline{pink}\tikzline{purple}
\tikzline{teal}\tikzline{violet}
还有white\tikz{\draw[white,line width=9](0,0)--(0.5,0);}.

\subsection{箭头}
箭头也是需要经常绘制的内容。主要也是根据\verb+\draw+命令的参数控制:

\noindent\begin{tabular}{p{0.25\linewidth}l}
\begin{tikzpicture}[baseline=(current bounding box.east)]
  \draw[->] (0.5,2.5) -- (2,3);
  \draw[<-] (0.5,1.5) -- (2,2);
  \draw[|->] (0.5,0.5)-- (2,1); 
  \draw[<->] (0,3) -- (0,0) -- (2,0);
\end{tikzpicture}
&
\begin{tikzcode}{}
\begin{tikzpicture}
  \draw[->] (0.5,2.5) -- (2,3);
  \draw[<-] (0.5,1.5) -- (2,2);
  \draw[|->] (0.5,0.5)-- (2,1); 
  \draw[<->] (0,3) -- (0,0) -- (2,0);
\end{tikzpicture}
\end{tikzcode}
\end{tabular}

\section{高效书写}
\subsection{变量}
变量申请使用与\LaTeX 相同的语句:\verb+\newcommand+. 注意:变量需要以反斜杠开头,并与现有命令不重复。

\noindent\begin{tabular}{p{0.25\linewidth}l}
\begin{tikzpicture}[baseline=(current bounding box.east)]
  \draw [help lines](0,0) grid (2,3);
  \newcommand{\aaa}{1};
  \newcommand{\bbb}{3};
  \newcommand{\ccc}{2};
  \coordinate (pA) at (\aaa,0);
  \coordinate (pB) at (\ccc,\bbb);
  \coordinate (pC) at (0,\ccc);
  \draw[fill=red] (pA) -- (pB) -- (pC) -- (pA); 
\end{tikzpicture}
&
\begin{tikzcode}{}
\begin{tikzpicture}
  \draw [help lines](0,0) grid (2,3);
  \newcommand{\aaa}{1};
  \newcommand{\bbb}{3};
  \newcommand{\ccc}{2};
  \coordinate (pA) at (\aaa,0);
  \coordinate (pB) at (\ccc,\bbb);
  \coordinate (pC) at (0,\ccc);
  \draw[fill=red] (pA) -- (pB) -- (pC) -- (pA); 
\end{tikzpicture}
\end{tikzcode}
\end{tabular}

\subsection{循环}
但是同样的语句书写很多遍是很复杂的,好在Tikz提供了循环的实现方法:

\noindent\begin{tabular}{p{0.25\linewidth}l}
\begin{tikzpicture}[baseline=(current bounding box.east)]
  \newcommand{\la}{1};
  \newcommand{\lb}{3};
  \newcommand{\lc}{2};
  \draw [help lines](0,0) grid (\lc,\lb);
  \coordinate (pA) at (\la,0);
  \coordinate (pB) at (\lc,\lb);
  \coordinate (pC) at (0,\lc);
  \tikzstyle{every node}=[circle, draw, fill=blue,inner sep=2pt];
  \foreach \x/\y in {A/270,B/0,C/180}{
    \node[label=\y:$\x$] at (p\x){};
  }
\end{tikzpicture}
&
\begin{tikzcode}{}
\begin{tikzpicture}
  \newcommand{\la}{1};
  \newcommand{\lb}{3};
  \newcommand{\lc}{2};
  \draw [help lines](0,0) grid (\lc,\lb);
  \coordinate (pA) at (\la,0);
  \coordinate (pB) at (\lc,\lb);
  \coordinate (pC) at (0,\lc);
  \tikzstyle{every node}=[circle, draw, fill=blue,inner sep=2pt];
  \foreach \x/\y in {A/270,B/0,C/180}{
    \node[label=\y:$\x$] at (p\x){};
  }
\end{tikzpicture}
\end{tikzcode}
\end{tabular}

注意到:甚至点的名称pA, pB, pC中的A, B, C也是可以通过foreach来指定的!

\subsection{运算}
\fbox{{\itshape{注意:需要在导言区添加}}\latexline{\\usetikzlibrary{calc}}。文中不再写出。}

如果一个绘图工具仅仅能够绘图而不能做运算……它有什么用呢?如果作为科学排版系统下的绘图工具而不支持运算的话,Tikz岂不是太弱了?

\noindent\begin{tabular}{p{0.25\linewidth}l}
\begin{tikzpicture}[baseline=(current bounding box.east)]
  \draw [help lines](0,0) grid (2,3);
  \coordinate (pA) at (1,0);
  \coordinate (pB) at (2,3);
  \coordinate (pC) at (0,2);
  \coordinate (pD) at ($(pB)+(0,-1)$);
  \tikzstyle{every node}=[circle, draw, fill=blue,inner sep=2pt];
  \foreach \x/\y in {A/270,B/0,C/45,D/315}{
    \node[label=\y:$\x$] at (p\x){};
  }
\end{tikzpicture}
&
\begin{tikzcode}{}
\begin{tikzpicture}
  \draw [help lines](0,0) grid (2,3);
  \coordinate (pA) at (1,0);
  \coordinate (pB) at (2,3);
  \coordinate (pC) at (0,2);
  \coordinate (pD) at ($(pB)+(0,-1)$);
  \tikzstyle{every node}=[circle, draw, fill=blue,inner sep=2pt];
  \foreach \x/\y in {A/270,B/0,C/45,D/315}{
    \node[label=\y:$\x$] at (p\x){};
  }
\end{tikzpicture}
\end{tikzcode}
\end{tabular}