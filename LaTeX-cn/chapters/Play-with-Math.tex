%!TEX root = ../LaTeX-cn.tex
\chapter{数学排版}
\section{行间与行内公式}
\co{行内公式}指将公式嵌入到文段的排版方式,主要要求公式垂直距离不能过高,否则影响排版效果。行内公式的书写方式:
\begin{latex}
$...$ 或者 \(...\) 或者 \begin{math}...\end{math}
\end{latex}

一般推荐第一或第二种方式。例:\verb|$\sum_{i=1}^{n}a_i$|,即:$\sum_{i=1}^{n}a_i$.

另外一种公式排版方式是\co{行间公式},也称行外公式,使用:
\begin{latex}
\[...\] 或者 \begin{displaymath}...\end{displaymath}
或者 amsmath 提供的 \begin{equation*}...\end{equation*}
\end{latex}

一般也推荐第一种命令\footnote{还有一种\texttt{\$\$...\$\$}的写法,源自底层\TeX,不建议使用。},例如:\verb|\[\sum_{i=1}^n{a_i}\]|,得到:
\[\sum_{i=1}^{n}a_i\]

从上面的两个例子可以看出,即使输出相同的内容,行内和行间的排版也是有区别的,比如累计符号上标是写在正上方还是写在右上角。

如果行间公式需要编号,使用\envi{equation}环境\footnote{需要注意有一个\RED{已被放弃}的多行公式编号环境叫\texttt{eqnarray},请不要再使用。},还可以插入标签:

\begin{codeshow}
\begin{equation}
\label{eq:NoExample}
  |\epsilon|>M
\end{equation}
\end{codeshow}

\section{数学字体、字号与空格}
\label{sec:mathfont}
\subsection{空格}
在数学环境中,行文空格是被忽略的。比如\verb|$x,y$|和\verb|$x, y$|并没有区别。数学环境有独有的空格命令,最后一个是$-3/18$的空格:

\begin{codeshow}
  $没有空格,3/18空\,格$ \\
  $4/18空\:格,5/18空\;格$ \\
  $9/18空\ 格,一个空\quad 格$ \\
  $两个空\qquad 格,负3/18空\!格$
\end{codeshow}

事实上,以上命令也可以在数学模式外使用,其中使用最广泛的是\latexline{,},比如上文提到过的千位分隔符。在数学环境中它也应用广泛(含有隐式的\latexline{,}):

\begin{codeshow}
\[ \int_0^1 x \ud{}x
= \frac{1}{2} \]
\end{codeshow}

其中\verb|\ud|命令是自定义的,这也是微分算子的正常定义\label{cmd:ud}:
\begin{latex}
\newcommand{\ud}{\mathop{}\negthinspace\mathrm{d}}
\end{latex}

\subsection{间距}
命令\latexline{abovedisplayskip}和\latexline{belowdisplayskip}控制了行间公式与上下文之间的间距,并且该值不会随字号调整而调整。有时你需要自行指定。默认值\texttt{12pt plus 3pt minus 9pt}。多行公式之间的间距用\latexline{jot}来控制,默认\texttt{3pt}。命令\latexline{mathsurround}给出了行内公式与文字间,除了预留空格之外的间距,默认值为\texttt{0pt}。另外一个有趣的命令\latexline{smash},可以忽略参数的全高:
\begin{codeshow}
\[\underline{\smash{\int f(x)\ud x}}=1\]
\end{codeshow}

也能够通过参数,单独忽略参数的高度(t)或深度(b):
\begin{codeshow}
$\sqrt{A_{n_k}} \qquad
\sqrt{\smash[b]{A_{n_k}}}$
\end{codeshow}

\subsection{字号}
\LaTeX\ 提供四种字号尺寸命令:
\begin{para}
\item[\latexline{displaystyle}] 行间公式尺寸。如$\displaystyle \sum_{i=1}^n a$
\item[\latexline{textstyle}] 行内公式尺寸。如$\textstyle \sum_{i=1}^n a$
\item[\latexline{scriptstyle}] 上下标尺寸。如$\scriptstyle \sum_{i=1}^n a$
\item[\latexline{scriptscriptstyle}] 次上下标尺寸。如$\scriptscriptstyle \sum_{i=1}^n a$
\end{para}

\subsection{数学字体}
将字体转为正体使用\latexline{mathrm}命令。如需保留空格,使用\latexline{textrm}命令——这与正文一致。但是,\latexline{textrm}命令内的字号可能不会自适应,\latexline{mathrm}则表现起来稳定得多。

例如自然对数的底数$\ue$,在本文中就是这样定义的:
\begin{latex}
\newcommand{\ue}{\mathrm{e}}
\end{latex}

以下简单介绍几种数学字体。数学字体的总表参见\tref{tab:mathfont}。

\subsubsection{数学粗体}
数学粗体使用\pkg{amsmath}宏包支持的\latexline{boldsymbol}命令。命令\latexline{boldmath}的问题在于它只能加粗一个数学环境,其中很可能包括了标点符号,而这是不严谨的。命令\latexline{mathbf}就差的更远,它只能把字体转为\textbf{正}粗体,而数学字体都是斜体的。

\begin{codeshow}
$\mu,M$\\ $\boldsymbol{\mu},
\boldsymbol{M}$
\end{codeshow}

\subsubsection{空心粗体}
空心粗体使用\pkg{amsfonts}或\pkg{amssymb}宏包的\latexline{mathbb}命令。这里用\latexline{textrm}而不是\latexline{mathrm},是为了保留空格。

\begin{codeshow}
$x^2 \geq 0 \qquad
\textrm{for all }x\in\mathbb{R}$
\end{codeshow}

\section{基本命令}
基本函数默认用正体书写,包括:
\begin{verbatim}
\sin \cos \tan \cot \arcsin \arccos \arctan \cot \sec \csc
\sinh \cosh \tanh \coth \log \lg \ln \ker \exp \dim \arg \deg 
\lim \limsup \liminf \sup \inf \min \max \det \Pr \gcd
\end{verbatim}

以上函数,最后一行的10个是可以带上下限参数的,即在行间公式模式下,上标和下标将在函数正上方和正下方书写内容。

\pkg{amsmath}宏包允许\latexline{DeclareMathOperator}命令自定义基本函数,用法类似于\latexline{newcommand}命令。如果命令带星号\latexline{DeclareMathOperator*},则可以带上下限参数。

此外有一个叫\latexline{mathop}的命令,可以把参数转换为数学对象,使其能够堆叠上下标;\latexline{mathbin}与\latexline{mathrel}则分别能把参数转换为二元运算符、二元关系符,并正确设置两侧的空距。

\subsection{上下标与虚位}
用低划线和尖角符表示上标和下标,请仔细体会下述例子:

\begin{codeshow}
$a^3_{ij}$ \\
${a_{ij}}^3\text{或}a_{ij}{}^3$\\
$\mathrm{e}^{x^2}\geq 1$
\end{codeshow}

上面的指数3的位置读者可以多多体会一下。此外,\latexline{phantom}被称为虚位命令,从下例你也能够体会到他的作用:

\begin{codeshow}
${}^{12}_{6}\mathrm{C}$ \\
${}^{12}_{\phantom{1}6}
\mathrm{C}$ \\
$a^3_{ij}$ \\
$a^{\phantom{ij}3}_{ij}$
\end{codeshow}

宏包\pkg{mathtools}提供了\latexline{prescript}来避免手工调整:
\begin{codeshow}
$\prescript{12}{6}{\mathrm{C}}$
\end{codeshow}

\subsection{微分与积分}
导数直接使用单引号\verb|'|,积分使用\latexline{int}符号:

\begin{codeshow}
$y'=x \qquad \dot{y}(t)=t$ \\
$\ddot{y}(t)=t+1$
$\dddot{y}+\ddddot{y}=0$ \\
$\iint_{D}f(x)=0$
$\int_{0}^{1}f(x)=1$
\end{codeshow}

有时候需要更高级的微分或积分号,其中\latexline{ud}命令在\hyperref[cmd:ud]{上文这里}定义过:
\begin{codeshow}
\[\left.\frac{\ud y}{\ud x}\right|_{x=0}\quad
\frac{\partial f}{\partial x}
\quad\oint\;\varoiint_S \]
\end{codeshow}

其中的\latexline{dot}系的导数形式\LaTeX\ 只原生支持到二阶导数。后面的三阶、四阶需要\pkg{amsmath}宏包。\latexline{int}系的积分命令类似。而环形双重积分命令\latexline{varoiint}需要\pkg{esint}宏包\footnote{该宏包可能与\pkg{amsmath}冲突,即便使用也请其放在\pkg{amsmath}之后加载。}。

\latexline{left.}或\latexline{right.}命令\footnote{参考\hyperref[subsec:delimiter]{定界符}部分的内容。}只用于匹配,本身不输出任何内容。

\subsection{分式、根式与堆叠}
分式使用\latexline{frac}命令。或者\pkg{amsmath}宏包支持的\latexline{dfrac}、\latexline{tfrac}命令来强制获得行间公式、行内公式大小的分数。如果想自定义分式样式,参考\secref{subsec:binom}一节的\latexline{genfrac}命令。
\begin{codeshow}
\[\frac{x}{y}+\dfrac{x}{y}
+\tfrac{a}{b}\]
\end{codeshow}

该宏包还支持另一个命令\latexline{cfrac},用于输入连分式。
\begin{codeshow}
\[\cfrac{1}{1+\cfrac{2}{1+x}}\]
\end{codeshow}

空根式用\latexline{surd}输出,更常用的是\latexline{sqrt}:
\begin{codeshow}
$\sqrt{2} \qquad \surd$\\
$\sqrt[\beta]{k}$
\end{codeshow}

开方次数的位置可以用这两个命令微调,参数是整数:
\begin{codeshow}
$\sqrt[\leftroot{-2}\uproot{2} \beta]{k}$
\end{codeshow}

划线命令使用\latexline{underline}和\latexline{overline},水平括号使用brace或者bracket代替line,例如\latexline{underbrace}:

\begin{codeshow}
$\overline{m+n}$ \\
$\underbrace{a_1+\ldots+a_n}_{n}$
$\overbrace{a_1+\ldots+a_n}^{n}$
% 可选参数:线宽;竖直空距
$\underbracket[0.4pt][1ex]
  {a_1+\cdots+a_n}_n$
\end{codeshow}

两个互有重叠的括号需要一个箱子命令\latexline{rlap},会在后面提到。不过在$j$之前的空距有些异常,可能需要\latexline{,}进行修正。
\begin{codeshow}
\[b+\rlap{$\overbrace{\phantom{
  c+d+e+f+g}}^x$}c+d+\underbrace{
  e+f+g+h+i}_y+\,j \]
\end{codeshow}

事实上\latexline{overline}命令也存在问题,请比较:

\begin{codeshow}
$\overline{A}\overline{B}$ \\
$\closure{A}\closure{B}$
$\closure{AB}$
\end{codeshow}

其中\latexline{closure}是在导言区定义的:
\begin{latex}
\newcommand{\closure}[2][3]{{}\mkern#1mu
    \overline{\mkern-#1mu#2}}
\end{latex}

还可以输出能堆叠到其他对象上的箭头符,比如向量符号:

\begin{codeshow}
  $\vec a\quad\overrightarrow{PQ}$
  $\overleftarrow{EF}$
\end{codeshow}

你也许还需要能够添加上下堆叠的箭头符:

\begin{codeshow}
\[ a\xleftarrow{x+y+z} b \]
\[ c\xrightarrow[x<y]{a*b*c}d \]
\end{codeshow}

尖帽符号、波浪符号,还有\pkg{yhmath}宏包支持的圆弧符号:

\begin{codeshow}
$\hat{A}\quad\widehat{AB}$\\
$\tilde{C}\quad\widetilde{CD}
\qquad\wideparen{APB}$
\end{codeshow}

强制堆叠命\latexline{stackrel},位于上方的符号与上标同等大小。如果有\pkg{amsmath}宏包,可以使用\latexline{overset}或者\latexline{underset} 命令,前者与\latexline{stackrel}命令完全等同:

\begin{codeshow}
$\int f(x) \stackrel{?}{=} 1$\\
$A\overset{abc}{=}B$ \quad $C\underset{def}{=}D$
\end{codeshow}

一个很强大的堆叠放置命令\latexline{sideset},只用于巨算符:

\begin{codeshow}
\[\sideset{_a^b}{_c^d}\sum\]
\[\sideset{}{'}\sum_{n=1}\text{或}
\,{\sum\limits_{n=1}}'\]
\end{codeshow}

去心邻域\latexline{mathring}大概也属于堆叠符的一种?这样输出:

\begin{codeshow}
$\mathring{U}$
\end{codeshow}

在下一次节:累加与累积中,还介绍了更多的堆叠命令。

\subsection{累加与累积}
使用\latexline{sum}和\latexline{prod}命令,效果如下:

\begin{codeshow}
\[\sum_{i=1}^{n}a_i=1 \qquad
\prod_{j=1}^{n}b_j=1\]
\end{codeshow}

有时需要复杂的堆叠方式,效果如下:

\begin{codeshow}
\[\sum_{\substack{0<i<n \\
  0<j<m}} p_{ij}=
  \prod_{\begin{subarray}{l}
  i\in I \\  1<j<m
  \end{subarray}}q_{ij}\]
\end{codeshow}

有时候需要强制实现堆叠的效果,可以使用\latexline{limits}命令。如果堆积目标不是数学对象,还需要使用\latexline{mathop}命令:

\begin{codeshow}
\[\max\limits_{i>1}^{x}\quad
\mathop{xyz}\limits_{x>0}\quad
\lim\nolimits_{x\to \infty}\]
\end{codeshow}

\subsection{矩阵与省略号}
矩阵的排版可以通过\envi{array}环境和自适应定界符完成:

\begin{codeshow}
\[\mathbf{A}=
\left(\begin{array}{ccc}
x_{11} & x_{12} & \ldots \\
x_{21} & x_{22} & \ldots \\
\vdots & \vdots & \ddots
\end{array}\right)\]
\end{codeshow}

还有就是\latexline{cdots}命令。\pkg{mathdots}宏包支持省略号缩放,并提供了\latexline{iddots}:$\iddots$. 或许什么时候需要使用呢?

\LaTeX\ 原生支持自动添加定界符的形式,不过要放在数学环境内:
\begin{codeshow}
\centering $\begin{matrix}
0 & 1 \\ 1 & 0 \end{matrix}\qquad
\begin{pmatrix} 0 & 2 \\
2 & 0 \end{pmatrix}$
\end{codeshow}

方括号和花括号使用\latexline{[Bb]matrix}命令:
\begin{codeshow}
\centering $\begin{bmatrix}
0 & 3 \\ 3 & 0 \end{bmatrix}\qquad
\begin{Bmatrix} 0 & 4 \\
4 & 0 \end{Bmatrix}$
\end{codeshow}

行列式使用\latexline{[Vv]matrix}命令:
\begin{codeshow}
\centering $\begin{vmatrix}
0 & 5 \\ 5 & 0 \end{vmatrix}\qquad
\begin{Vmatrix} 0 & 6 \\
6 & 0 \end{Vmatrix}$
\end{codeshow}

宏包\pkg{mathtools}的带星\texttt{matrix}命令,可更改列对齐:
\begin{codeshow}
$\begin{pmatrix*}[r]
100 & -200 \\ 20 & 10
\end{pmatrix*}$
\end{codeshow}

在矩阵中排版\latexline{dfrac}分式时,处理行距如下例的 \texttt{\char92\char92 [8pt]}:
\begin{codeshow}
\[\mathbf{H}=\begin{bmatrix}
\dfrac{\partial^2 f}{\partial x^2} &
\dfrac{\partial^2 f}
{\partial x \partial y} \\[8pt]
\dfrac{\partial^2 f}
{\partial x \partial y} &
\dfrac{\partial^2 f}{\partial y^2}
\end{bmatrix}\]
\end{codeshow}

宏包\pkg{amsmath}还支持行内小矩阵\latexline{smallmatrix},需手动加括号。
\begin{codeshow}
矩阵 $\left(\begin{smallmatrix}
x & -y\\ y & x\end{smallmatrix}
\right)$ 可以显示在行内。
\end{codeshow}

最后,一种带边注的矩阵\latexline{bordermatrix},用法有些奇怪:
\begin{codeshow}
\[\bordermatrix{& 1 & 2\cr
1 & A & B \cr
2 & C & D \cr} \]
\end{codeshow}

\subsection{分段函数与联立方程}
用\envi{cases}环境书写分段函数,它自动生成一个比\latexline{left\{}更紧凑的花括号:

\begin{codeshow}
\[y=\begin{cases}
\int x, & x>0 \\
0,   & x=0 \\
x-1, & x<0
\end{cases},\,x\in\mathbb{R}\]
\end{codeshow}

如果想要生成display样式的内容(比如上面的积分号只是text样式的),使用\pkg{mathtools}宏包的\envi{dcases}环境代替\envi{cases}环境。如果\envi{cases}环境的第二列条件不是数学语言而是一般文字,可以考虑使用\envi{dcases*}环境,列中用\&{}隔开。

\begin{codeshow}
\[y=\begin{dcases}
  \int x, & x>0 \\
  x^2, & x\leqslant 0
  \end{dcases}\]
\[z=\begin{dcases*}
  y, & when $y$ is prime\\
  y^2, & otherwise
  \end{dcases*}\]
\end{codeshow}

\subsection{多行公式及其编号}
\label{subsec:multieqnum}
多行公式可以使用\pkg{amsmath}下的\envi{align}环境——因为原生的\envi{eqnarray}环境真的很差!而且\envi{align}环境不需要像\envi{array}环境那样给出列的数目和参数,能够根据
\texttt{\&}符号的数量来自调整。\qd{这个环境会自动对齐等号或者不等号,所以必要时请用\&指定对齐位置}。下面是一个例子:

\begin{codeshow}
\begin{align}
  a^2  &= a\cdot a \\
       &= a*a      \\
       &= a^2
\end{align}
\end{codeshow}

\LaTeX\ 中长公式不能自动换行\footnote{不过\pkg{breqn}宏包的\envi{dmath}环境可以实现自动换行,读者可以自行尝试效果。},请如上自行指定断行位置和缩进距离。

至于多行公式换页,可以在导言区加上\latexline{allowdisplaybreaks}实现(可选参数:1为尽量避免换页,2至4为倾向于换页),或在特定位置加上\latexline{displaybreak}(可选参数:0为允许在下个换行符后换页,但不倾向换页;2-3介中;4为强制换页)。两种的默认可选参数都是4。

上例给出三个编号,如果你只需要一个,可以:

\begin{codeshow}
\begin{align}
  a^2&= a\cdot a& b&=c\nonumber\\
  g  &= a*a & d&>e>f  \nonumber\\
  step&= a^2 & &Z^3
\end{align}
\end{codeshow}

如果你想让编号显示在这三行的中间而不是最下面一行,可以尝试把公式写在\envi{aligned}或者\envi{gathered}环境中,然后再嵌套到\envi{equation}环境内。如果你根本不想给多行公式编号,尝试\envi{align*}环境。

另外,\pkg{amsmath}宏包的\envi{multline}环境将自动把编号放在末行。首行左对齐,末行右对齐,中间的行居中。
\begin{codeshow}
\begin{multline}
a>b \\
b>c \\
\therefore a>c
\end{multline}
\end{codeshow}

如果想在环境中插入小段行间文字,使用\latexline{intertext}命令,或者\pkg{mathtools}宏包的\latexline{shortintertext}命令。区别是后者的垂直间距更小一些。

\begin{codeshow}
\begin{align*}
\shortintertext{If}
 y &= 0 \\
 x &< 0\\
\shortintertext{then}
 z &= x+y
\end{align*}
\end{codeshow}

当然,\envi{align}环境用于分列对齐的。如果仅想所有行居中,使用\pkg{amsmath}宏包的\envi{gather}环境即可。这是一个非常实用的环境,你也可以用\envi{gather*}环境排版居中的、非编号的多行公式。

\begin{codeshow}
\begin{gather}
  X=1+2+\cdots+n \\
  Y=1
\end{gather}
\end{codeshow}

\subsection{二项式}
\label{subsec:binom}
二项式可能需要借助\pkg{amsmath}宏包的\latexline{binom}命令。它也有像分式一样的行间和行内两个命令\latexline{tbinom}与\latexline{dbinom}:

\begin{codeshow}
$\mathrm{C}_n^k=\binom{n}{k}
\qquad a_n=\dbinom{n}{k}$
\end{codeshow}

你也可以通过该宏包支持的\latexline{genfrac}自定义类似二项式命令:
\begin{latex}
\genfrac{left-delim}{right-delim}{thickness}{mathstyle}
{numerator}{denominator}
% thickness为分式线线宽,留空空表示默认
% mathstyle从0-3由\displaystyle减至\scriptscriptstyle
\newcommand{\Bfrac}[2]{\genfrac{[}{]}{0pt}{}{#1}{#2}}
\end{latex}

你可以借此得到新的命令\latexline{Bfrac}:
\begin{codeshow}
\[\text{We define } \Bfrac{n}{k} = \binom{k}{n}\]
\end{codeshow}

\subsection{定理}
在使用下述定理内容时,请加载\pkg{amsthm}宏包。

首先是定理环境格式的自定义。如同定义命令一样,在导言区加上:
\begin{latex}
\newtheorem{envname}[counter]{text}[section]
\end{latex}

其中\textit{name}表示定理的引用名称,即下文将其作为一个环境名来识别;\textit{text}表示定理的显示名称,即下文中定理将以其作为打印内容。而\textit{counter}参数表示你是否与先前声明的某定理共同编号。\textit{section}参数表示定理的计数层级,如果是section,表示每节分别计数;chapter表示每章分别计数。

来看一个例子。首先在导言区定义如下三个样式:
\begin{latex}
\theoremstyle{definition}\newtheorem{laws}{Law}[section]
\theoremstyle{plain}\newtheorem{ju}[laws]{Jury}
\theoremstyle{remark}\newtheorem*{marg}{Margaret}
\end{latex}

以上三个\latexline{theoremstyle}即是它预定义的所有样式类型。definition标题粗体,内容罗马体;plain标题粗体,内容斜体;remark标题斜体,内容罗马体。带星号表示不进行计数。在环境的使用中可以添加可选参数,用于以括号的形式注释定理。然后这是示例:

\begin{codeshow}
\begin{laws}
Never believe easily.
\end{laws}
\begin{ju}[The 2nd]
Never suspect too much.
\end{ju}
\begin{marg}Nothing else.\end{marg}
\end{codeshow}

\pkg{amsthm}宏包还提供了\envi{proof}环境,并且用\latexline{qedhere}来指定证毕符号的位置。如果不加指定,将会自动另起一行。

\begin{codeshow}
\begin{proof}
For an right triangle, we have:
  \[a^2+b^2=c^2 \qedhere\]
\end{proof}
\end{codeshow}

\section{数学符号与字体}
\subsection{数学字体}
原生的数学字体命令:
\begin{center}
\begin{minipage}{\linewidth}
\centering
\tabcaption{原生数学字体表}
\label{tab:mathfont}
\begin{tabular}{>{\ttfamily\char92}l>{$}l<{$}}
\hline
mathrm\{ABCDabcde 1234\} & \mathrm{ABCDabcde 1234} \\
\hline
mathit\{ABCDabcde 1234\} & \mathit{ABCDabcde 123} \\
\hline
mathnormal\{ABCDabcde 1234\} & \mathnormal{ABCDabcde 1234} \\
\hline
mathcal\{ABCDabcde 1234\} & \mathcal{ABCDabcde 1234} \\
\hline
\end{tabular}
\end{minipage}
\end{center}

需要其他宏包支持的数学字体:
\begin{center}
\begin{minipage}{\linewidth}
\centering
\tabcaption{宏包数学字体表}
\label{tab:mathfont-pk}
\begin{tabular}{>{\ttfamily}ll}
\hline
\char92mathscr\{ABCDabcde 1234\} & mathrsfs\\
$\mathscr{ABCDabcde 1234}$ & \\
\hline
\char92mathfrak\{ABCDabcde 1234\} & amsfonts或者amssymb\\
$\mathfrak{ABCDabcde 1234}$ & \\
\hline
\char92mathbb\{ABCDabcde 1234\} & amsfonts或者amssymb\\
$\mathbb{ABCDabcde 1234}$ & \\
\hline
\end{tabular}
\end{minipage}
\end{center}

\subsection{定界符}
\label{subsec:delimiter}
\tref{tab:delimiter}给出了一些数学环境中使用的定界符。

\begin{table}[!htb]
\centering
\caption{定界符}
\label{tab:delimiter}
\begin{tabular}{@{}*{3}{>{$}p{2em}<{$} @{} >{\ttfamily}p{7em}}}
( & ( & [ & [ or \char92 lbrack & \uparrow & \char92 uparrow \\
) & ) & ] & ] or \char92 rbrack & \downarrow & \char92 downarrow \\
\{ & \{ or \char92 lbrace & \} & \} or \char92 rbrace & \updownarrow & \char92 updownarrow \\
\langle & \char92 langle & \rangle & \char92 rangle & \backslash & \char92 backslash \\
\lfloor & \char92 lfloor & \rfloor & \char92 rfloor & \Updownarrow & \char92 Updownarrow \\
\lceil & \char92 lceil & \rceil & \char92 rceil & \Uparrow & \char92 Uparrow \\
\Vert & \char92 | or \char92 Vert & | & | or \char92 vert & \Downarrow & \char92 Downarrow \\
\hline
\multicolumn{6}{c}{-- 以下需要amssymb宏包 --} \\
\multicolumn{3}{c}{$\ulcorner$ \quad \texttt{\char92 ulcorner}} & \multicolumn{3}{c}{$\urcorner$ \quad \texttt{\char92 urcorner}} \\
\multicolumn{3}{c}{$\llcorner$ \quad \texttt{\char92 llcorner}} & \multicolumn{3}{c}{$\lrcorner$ \quad \texttt{\char92 lrcorner}}
\end{tabular}
\end{table}

使用\latexline{left}, \latexline{right}还有\latexline{middle}能够使定界符自适应式子的高度:
\begin{codeshow}
\[P\left(X \middle\vert Y=0\right)
=\left.\int_0^1 p(t)\ud t\middle/ N\right.\]
\end{codeshow}

如果希望手动指定定界符的尺寸,这时使用后:
\begin{codeshow}
% 加l, r, m对应上述三种自适应命令
$(\big(\Big(\bigg(\Bigg<\qquad
\bigl[\frac{x+y}{x^2}\bigr]$
\end{codeshow}

有时\latexline{left.}和\latexline{right.}能灵活地用于跨行控制,因为它们并非实际配对:
\begin{codeshow}
\begin{align*}
  x &=\left(\frac{1}{2}x\right.\\
  &\left.\vphantom{\frac{1}{2}}
  +y^2+z_1\right)
\end{align*}
\end{codeshow}

其中\latexline{vphantom}命令用于输出一个高度虚位,使得第二行的自适应定界符与第一行同等大小。特别地,命令\latexline{mathstrut}表示一个有圆括号总高的虚位:
\begin{codeshow}
$\sqrt{b}\sqrt{y}\qquad
\sqrt{\mathstrut b}\sqrt{\mathstrut y}$
\end{codeshow}

\subsection{希腊字母}
希腊字母表如\tref{tab:greekletter}所示。表中包含了小写希腊字母、大写希腊字母,其中部分希腊字母的输入方式与英文字母一致。
\begin{table}[!htb]
\centering
\caption{希腊字母表}
\label{tab:greekletter}
\renewcommand\arraystretch{1}
\begin{tabular}{*{4}{>{$}c<{$}!{}>{\ttfamily\char`\\}p{6em} @{}}}
\alpha & alpha & \theta & theta & o & \multicolumn{1}{p{6em}}{o} & \upsilon & upsilon \\
\beta & beta & \vartheta & vartheta & \pi & pi & \phi & phi \\
\gamma & gamma & \iota & iota & \varpi & varpi & \varphi & varphi \\
\delta & delta & \kappa & kappa & \rho & rho & \chi & chi \\
\epsilon & epsilon & \lambda & lambda & \varrho & varrho & \psi & psi \\
\varepsilon & varepsilon & \mu & mu & \sigma & sigma & \omega & omega \\
\zeta & zeta & \nu & nu & \varsigma & varsigma & \eta & eta \\
\xi & xi & \tau & tau & \multicolumn{4}{c}{} \\
A & \multicolumn{1}{p{6em}}{A} & B & \multicolumn{1}{p{6em}}{B} & \Gamma & Gamma & \varGamma & varGamma \\
\Delta & Delta & \varDelta & varDelta & E & \multicolumn{1}{p{6em}}{E} & Z & \multicolumn{1}{p{6em}}{Z} \\
H & \multicolumn{1}{p{6em}}{H} & \Theta & Theta & \varTheta & varTheta & I & \multicolumn{1}{p{6em}}{I} \\
\Lambda & Lambda & \varLambda & varLambda & M & \multicolumn{1}{p{6em}}{M} & N & \multicolumn{1}{p{6em}}{N} \\
\Xi & Xi & \varXi & varXi & O & \multicolumn{1}{p{6em}}{O} & \Pi & Pi \\
\varPi & varPi & P & \multicolumn{1}{p{6em}}{P} & \Sigma & Sigma & \varSigma & varSigma \\
T & \multicolumn{1}{p{6em}}{T} & \Upsilon & Upsilon & \varUpsilon & varUpsilon & \Phi & Phi \\
\varPhi & varPhi & X & \multicolumn{1}{p{6em}}{X} & \Psi & Psi & \varPsi & varPsi \\
\Omega & Omega & \varOmega & varOmega & \multicolumn{4}{c}{}
\end{tabular}
\end{table}

\subsection{二元运算符}
二元运算符包括常见的加减乘除,还有集合的交、并、补等运算。\tref{tab:operator}只列出常用的二元运算符,更多的请参考symbols-a4文档。
\begin{table}[!htb]
\centering
\caption{二元运算符:\latexline{mathbin}}
\label{tab:operator}
\renewcommand\arraystretch{1}
\begin{tabular}{@{}*{2}{>{$}c<{$}!{} >{\ttfamily\char92}p{5em} @{}}*{2}{>{$}p{2em}<{$} @{} >{\ttfamily\char92}p{6em} @{}}}
+ & \multicolumn{1}{p{6em}}{+} & - & \multicolumn{1}{p{6em}}{-} & \times & times & \div & div \\
\pm & pm & \mp & mp & \circ & circ & \triangleright & triangleright \\
\cdot & cdot & \star & star & \ast & ast & \triangleleft & triangleleft \\
\cup & cup & \cap & cap & \setminus & setminus & \bullet & bullet \\
\oplus & oplus &\ominus & ominus & \otimes & otimes & \oslash & oslash \\
\odot & odot & \bigcirc & bigcirc & \vee & vee,lor & \wedge & wedge,land \\
\bigcup & bigcup & \bigcap & bigcap & \bigvee & bigvee & \bigwedge & bigwedge
\end{tabular}
\end{table}

\subsection{二元关系符}
二元关系符常常被用于判断两个数的大小关系,或者集合中的从属关系。\tref{tab:relation-operator}和\tref{tab:amsrelation-operator}只列出常用的二元关系符,更多的请参考symbols-a4文档。
\begin{table}[!htb]
\centering
\caption{二元关系符:\latexline{mathrel}}
\label{tab:relation-operator}
\renewcommand\arraystretch{1}
\begin{tabular}{@{}*{4}{>{$}c<{$}!{} >{\ttfamily\char92}p{6em} @{}}}
< & \multicolumn{1}{p{6em}}{<} & > & \multicolumn{1}{p{6em}}{>} & \le & le(q) & \ge & ge(q) \\
\ll & ll & \gg & gg & \equiv & equiv & \neq & neq \\
\prec & prec & \succ & succ & \preceq & preceq & \succeq & succeq \\
\sim & sim & \simeq & simeq & \cong & cong & \approx & approx \\
\subset & subset & \supset & supset & \subseteq & subseteq & \supseteq & supseteq \\
\in & in & \ni & ni & \notin & notin & \propto & propto \\
\parallel & parallel & \perp & perp & \smile & smile & \frown & frown \\
\asymp & asymp & \bowtie & bowtie & \vdash & vdash & \dashv & dashv
\end{tabular}
\end{table}

\tref{tab:amsrelation-operator}中的二元关系符需要\pkg{amssymb}宏包。
\begin{table}[!htb]
\centering
\caption{amssymb二元关系符}
\label{tab:amsrelation-operator}
\renewcommand\arraystretch{1}
\begin{tabular}{@{}*{4}{>{$}c<{$}!{} >{\ttfamily\char92}p{6em} @{}}}
\leqslant & leqslant & \geqslant & geqslant & \because & because & \therefore & therefore \\
\nless & nless & \ngtr & ngtr & \lessdot & lessdot & \gtrdot & gtrdot \\
\lessgtr & lessgtr & \gtrless & gtrless & \lesseqqgtr & lesseqqgtr & \gtreqqless & gtreqqless \\
\subseteqq & subseteqq & \supseteqq & supseteqq & \subsetneqq & subsetneqq & \supsetneqq & supsetneqq
\end{tabular}
\end{table}

\subsection{箭头与长等号}
在\tref{tab:delimiter}中给出了几个箭头符号,但是不够全,这里给出总表如\tref{tab:arrow}。
\begin{table}[!htb]
\centering
\caption{箭头}
\label{tab:arrow}
\renewcommand\arraystretch{1}
\begin{tabular}{@{}*{2}{>{$}c<{$}!{} >{\ttfamily\char92}p{10em} @{}}}
\leftarrow & leftarrow & \longleftarrow & longleftarrow \\
\rightarrow & rightarrow & \longrightarrow & longrightarrow \\
\leftrightarrow & leftrightarrow & \longleftrightarrow & longleftrightarrow \\
\Leftarrow & Leftarrow & \Longleftarrow & Longleftarrow \\
\Rightarrow & Rightarrow & \Longrightarrow & Longrightarrow \\
\Leftrightarrow & Leftrightarrow & \Longleftrightarrow & Longleftrightarrow \\
\mapsto & mapsto & \longmapsto & longmapsto \\
\nearrow & nearrow & \searrow & searrow \\
\swarrow & swarrow & \nwarrow & nwarrow \\
\leftharpoonup & leftharpoonup & \rightharpoonup & rightharpoonup \\
\leftharpoondown & leftharpoondown & \rightharpoondown & rightharpoondown \\
\rightleftharpoons & rightleftharpoons & \iff & iff (bigger space)
\end{tabular}
\end{table}

\LaTeX\ 定义了逻辑命令\latexline{iff}, \latexline{implies}, \latexline{impliedby},与箭头符大小相同但是两侧间距更大:
\begin{codeshow}
$x=y \implies a=b$\\
$x=y \impliedby a=b$\\
$x=y \iff a=b$
\end{codeshow}

依旧另外给出一个基于\pkg{amssymb}宏包的附\tref{tab:amsarrow}。
\begin{table}[!htb]
\centering
\caption{amssymb箭头}
\label{tab:amsarrow}
\renewcommand\arraystretch{1}
\begin{tabular}{@{}*{2}{>{$}c<{$}!{} >{\ttfamily\char92}p{10em} @{}}}
\dashleftarrow & dashleftarrow & \dashrightarrow & dashrightarrow \\
\circlearrowleft & circlearrowleft & \circlearrowright & circlearrowright \\
\leftrightarrows & leftrightarrows & \rightleftarrows & leftrightarrows \\
\nleftarrow & nleftarrow & \nLeftarrow & nLeftarrow \\
\nrightarrow & nrightarrow & \nRightarrow & nRightarrow \\
\nleftrightarrow & nleftrightarrow & \nLeftrightarrow & nLeftrightarrow
\end{tabular}
\end{table}

最后,宏包\pkg{extarrows}给出了一些实用的长箭头与长等符号:

\begin{codeshow}
$\xlongequal{\Delta}$\quad
$\xLeftrightarrow{\Delta}$\\
$\xleftrightarrow{x=\tan t}$\\
$\xLongleftarrow{x} \xLongrightarrow{y}$
\end{codeshow}

\subsection{其他符号}
注意冒号如果从键盘输入,会识别为关系符,例如$:=$。在表示比例时也可以借用,或者外加\latexline{mathbin}命令$a\mathbin{:}b$。数学中可能用到的冒号,请使用\latexline{colon}命令,像$x\colon y\to\infty$这样。

像\latexline{\-}一样在数学环境中使用\latexline{*}命令,提醒\LaTeX\ 此处可以断词。\LaTeX\ 如果在此处断词,会自动补一个$\times$符号。你也可以自定义:
\begin{latex}
\renewcommand{\*}{discretionaty{\,\mbox{$\cdot$}}{}{}}
\end{latex}

最后是一些其他的难以归类的符号,也不全是数学领域会用到的,只不过它们可以在数学环境下输出出来,以及被\pkg{amssymb}宏包所支持。如\tref{tab:othersym}和\tref{tab:amsothersym}。
\begin{table}[!htb]
\centering
\caption{其他符号}
\label{tab:othersym}
\renewcommand\arraystretch{1}
\begin{tabular}{@{}*{4}{>{$}c<{$}!{} >{\ttfamily\char92}p{5.5em} @{}}}
\dots & dots & \cdots &cdots &
\vdots & vdots & \ddots & ddots \\
\forall & forall & \exists & exists &
\Re & Re & \aleph & aleph \\
\angle & angle & \infty & infty &
\triangle & triangle & \nabla & nabla \\
\hbar & hbar & \imath & imath &
\jmath & jmath & \ell & ell \\
\spadesuit & spadesuit & \heartsuit & heartsuit &
\clubsuit & clubsuit & \diamondsuit & diamondsuit \\
\flat & flat & \natural & natural &
\sharp & sharp & \multicolumn{2}{l}{} \\
\hline
\multicolumn{8}{l}{非数学符号:} \\
\multicolumn{1}{p{2em}}{\pounds} & pounds & \multicolumn{1}{p{2em}}{\S} & S &
\multicolumn{1}{p{2em}}{\copyright} & copyright & \multicolumn{1}{p{2em}}{\P} & P \\
\multicolumn{1}{p{2em}}{\dag} & dag & \multicolumn{1}{p{2em}}{\ddag} & ddag &
\multicolumn{1}{p{2em}}{\textregistered} & \multicolumn{3}{l}{\ttfamily \char`\\ textregistered}
\end{tabular}
\end{table}

\begin{table}[!htb]
\centering
\caption{amssymb其他符号}
\label{tab:amsothersym}
\renewcommand\arraystretch{1}
\begin{tabular}{@{} >{$}c<{$}!{} >{\ttfamily\char92}p{6em} @{}*{2}{>{$}p{2em}<{$} @{} >{\ttfamily\char92}p{8em} @{}}}
\square & square & \blacksquare & blacksquare & \hslash & hslash \\
\bigstar & bigstar & \blacktriangle & blacktriangle & \blacktriangledown & blacktriangledown \\
\lozenge & lozenge & \blacklozenge & blacklozenge & \measuredangle & measuredangle \\
\mho & mho & \varnothing & varnothing & \eth & eth
\end{tabular}
\end{table}
