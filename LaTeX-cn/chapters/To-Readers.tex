%!TEX root = ../LaTeX-cn.tex
\chapter{\mbox{写给读者}*}

我见过许多朋友初试\LaTeX\ ,他们都感到非常不能理解.主要有以下几个疑问:
\begin{feae}
\item “我平常使用MS Word,似乎也能完成科技排版工作,那么为什么我还需要\LaTeX\ ?”\hfill \textit{——见“为什么需要\LaTeX\ ?”一节.}
\item “\LaTeX\ 看上去不像是排版工具,更像是编程语言.我讨厌用写代码一样的方式来写文章.”\hfill \textit{——文本文件使你更专注于内容而不是排版细节.}
\item “\LaTeX\ 能生成doc文件吗?我平时上交作业/提交汇报时难道使用不便修改pdf文件么?”\hfill \textit{——见“\LaTeX\ 生成的文件格式?”一节.}
\end{feae}

本章希望能解决读者的这些疑问,让读者对于\LaTeX\ 有基础的了解,再决定是否需要学习.当然,如果你是被迫进入了\LaTeX\ 这个坑,你也可以阅读本章,或许本章能让你喜欢上\LaTeX\ 呢.

\section{什么是\LaTeX\ ?}
先讲\TeX\ (读音类似于“泰赫”).

\TeX\ 是Knuth\footnote{Donald Ervin Knuth(高德纳,1938--),现代计算机科学的先驱者,斯坦福大学计算机系的终身荣誉教授,图灵奖和冯诺依曼奖得主,\TeX\ 和 \hologo{METAFONT} 的发明人.同时也是业内经久不衰的著作\emph{The Art of Computer Programming}的作者.}研发的免费、开源的排版系统,其初衷是为了“改变排版界糟糕的排版技术”,并用于排版他的著作《计算机程序设计艺术》.

\TeX\ 对于读者来说应该是底层的内容,如果你有兴趣,可以阅读Knuth著的\emph{The \TeX\ book},这本书是学习\TeX\ 最权威的材料,没有之一.在本手册的参考文献中也给出了其他\TeX\ 学习的资料.\dpar

再讲\LaTeX\ (读音类似于“拉泰赫”).

\LaTeX\ 是基于\TeX\ 的宏\footnote{“宏(Macro)”是一个计算机概念,指用单个命令或操作完成一系列底层命令或操作的组合.}集,其作者是Dr.~Lamport\footnote{Leslie Lamport(1941--),美国计算机科学家,图灵奖和冯诺依曼奖得主.};其姓氏开头两个字母La与底层排版系统\TeX\ 相结合,就组成了名称\LaTeX\ .\LaTeX\ 在\TeX\ 基础上定义了众多的宏命令,使得用户可以更方便地进行排版.本手册的参考文献中就有他的作品.

\LaTeX\ 现在的版本是\LaTeXe ,意思是2.x版而没到3.0版.错位排版的字母E和字母A暗示了它是排版系统.在无法这样输出的场合,请写作LaTeX和LaTeX~2e.

\section{\TeX\ 与\LaTeX\ 的优缺点}
\TeX\ 的优点:\qd{稳定、精确、美观}.底层的\TeX\ 系统已经很多年没有进行大的变动了,因为它注重\uline{稳定};\TeX\ 系统可以让你把排版内容通过数字参数的方式写到任意的位置,量化的参数意味着\uline{精确};\TeX\ 底层的空距调整机制,以及对于数学公式近乎完美的支持,则确保了排版效果的\uline{美观}.

\LaTeX\ 是基于\TeX\ 的,自然不会抛弃上述\TeX\ 的优点.具体包括:
\begin{feai}
\item 排版出来就是印刷品.专业而美观.
\item 易用、全面的数学排版支持,无出其右.
\item 撰写文档时不会被文档排版细节干扰精力.你可以使用之前自定义的模板,或者方便地在文字组织完毕后调整你的模板,以轻松达到满意的效果.
\item 复杂的排版功能支持,比如图表目录、索引、参考文献管理、高度自定义的目录样式、双栏甚至多栏排版.
\item 丰富的功能以及易寻的帮助文档.众多的\LaTeX\ 宏包赋予了\LaTeX\ 强大的扩展功能,它们都自带文档供你学习.
\item 源文件是\RED{文本文件}\footnote{文本文件的另一个优点是易于进行版本控制,比如利用git. 你可以方便地比较你相比上次修改了什么内容,也可以方便地恢复到之前某个时刻的版本.}.你可以在任何设备、任何文本编辑器中书写文档内容,无须担心复制时格式的变化;最后粘贴到同一个tex文件中编译即可.
\item 跨平台,免费,开源.
\end{feai}

那缺点呢?我认为主要有:
\begin{feai}
\item 入门门槛高.想要熟练地使用\LaTeX\ 并轻松地编写有自己的风格的文档,不是一两天就能够达到的.
\item 并非“所见即所得”,需要编译才能看到效果.编译查错有时令人恼火.
\item 完善一个自己的模板可能需要很长的时间.尽管\LaTeX\ 原生定义的模板能够满足绝大多数场合的需要.
\item 排版长表格有些复杂.但作为补充,在表格内插入数学公式是非常简单的.
\end{feai}

\section{为什么需要\LaTeX\ ?}
你可能基于以下原因学习\LaTeX\ :
\begin{feae}
\item 你的投稿对象要求你使用\LaTeX\ 排版,而不是MS Word——这种情况对没听说过\LaTeX\ 的你来说,真是糟糕透了.
\item 你需要在多个设备上撰写同一份文档.但你发现把内容在多个文档间复制粘贴时,格式总是会出现问题.
\item 你受够了MS Word自带的公式编辑器,或者你觉得购买的插件MathType的效果也不尽如人意.但你经常需要排版公式.
\item 你想参加某个科学竞赛,比如MCM,然后你发现你的朋友用的一个叫\LaTeX\ 的东西似乎还不错.
\item 你想出版一本书,或者投稿你的作品——结果他们告知你如果你使用\LaTeX\ 而不是MS Word攥写原稿件,他们会更快地把作品印刷出来.
\item 呃……也许,你只是喜欢学习新事物?
\end{feae}

对于科研工作者或者在校研究生,我认为\LaTeX\ 是非常优秀的工具.如果你是本科生,或者更年轻的群体,你也可以先学习\LaTeX\ ,因为到了研究生和工作中,学习这类基础工具的时间可能就非常有限了.

\section{MS Word难道不优秀吗?}
我想说的是,\qd{MS Word当然是优秀的软件}.但是它与\LaTeX\ 的定位不同,所以它们分别适用于不同场合.前者注重简单组织内容,后者注重排版效果.

在\uline{排版书籍、科学文档}方面,\LaTeX\ 非常专业、美观,公式支持性极佳,几乎所有参数你都可以量化调整.如果你想\uline{高度自定义一份文件},比如拥有特殊几何、颜色元素,且易于更改模板的简历,\LaTeX\ 可以完全独当一面.在这些这方面,MS Word是无法匹敌\LaTeX\ 的.

但是如果你只是为了生成\uline{非正式的文档},比如1--2页的作业稿;或者只是一份\uline{易于别人修改的非科学稿件},比如一份需要同事修改的演讲稿……那你无须使用\LaTeX\ .这些方面,\LaTeX\ 无疑是比不上MS Word的.

\section{\LaTeX\ 生成的文件格式?}
一般广为使用的是pdf,以及dvi的格式.\LaTeX\ 无法生成doc或者docx扩展名文件,因为那是属于MS的商用格式,两者的工作机理也完全不同.

所以很遗憾,如果你身处一个要求你“必须提交docx”的环境中,那么\LaTeX\ 对你并不是一个好选择.但我想指出的是,这是稳定、优秀的pdf格式没有得到你身处环境认可的遗憾——pdf也可以方便地添加批注,并在不同设备的显示上有更好的稳定性.