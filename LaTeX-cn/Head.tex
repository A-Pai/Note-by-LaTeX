% safe参数解决与\!在内的多个冲突
% \sups命令可能被重定义,xeCJK放在tipa后
\RequirePackage[safe]{tipa}
\documentclass[a4paper, zihao=-4, linespread=1]{ctexrep}
  \xeCJKsetup{CJKmath=true}
  \renewcommand{\CTEXthechapter}{\arabic{chapter}}


% 中文支持
% \iffalse与\fi可以实现“多行注释”
\iffalse
\usepackage[slantfont,boldfont]{xeCJK}
	\setCJKmainfont[BoldFont=SimHei,ItalicFont=KaiTi]{SimSun}
	\setCJKmathfont{STXinwei}
\usepackage{indentfirst}
\fi
\setCJKmathfont{STXinwei}
\newCJKfontfamily[zhxinwei]\xinwei{STXinwei}

% 数学环境
\usepackage{amsmath}
  \newcommand{\ue}{\mathrm{e}}
  \newcommand{\ud}{\mathop{}\negthinspace\mathrm{d}}
\usepackage{amssymb}
\usepackage{mathrsfs} % 线性代数字体
    % overline的替代命令
\newcommand{\closure}[2][3]{{}\mkern#1mu\overline{\mkern-#1mu#2}}
\usepackage{yhmath} % 左下-右上省略号
\usepackage{mathtools} % dcases环境
\usepackage{amsthm} % 定理环境
  \theoremstyle{definition}\newtheorem{laws}{Law}[section]
  \theoremstyle{plain}\newtheorem{ju}[laws]{Jury}
  \theoremstyle{remark}\newtheorem*{marg}{Margaret}
\usepackage{esint} % 多重积分,需放在amsmath后

% 下划线宏包
\usepackage{ulem}
% LaTeX符号宏包
\usepackage{hologo}
	\newcommand{\xelatex}{\Hologo{XeLaTeX}}
	\newcommand{\bibtex}{\Hologo{BibTeX}}
% 其他符号
\usepackage{wasysym}
% 带箱小页
\usepackage{boxedminipage}
% 绘图
\usepackage{tikz}
	\usetikzlibrary{calc}
	\newcommand{\tikzline}[1]{{#1\tikz{\draw[#1,line width=9](0,0)--(0.5,0);}}, }

% 奇怪的小定义
\newcommand{\dpar}{\\ \mbox{}}	% 空两行
\newcommand{\qd}[1]{{\bfseries{#1}}}	% 强调
\newcommand{\co}[1]{{\bfseries{#1}}}   % Style of concept
\newcommand{\RED}[1]{{\color{red}{#1}}}
\newcommand{\cmmd}[1]{\fbox{\texttt{\char92{}#1}}}
\newcommand{\charef}[1]{第\ref{#1}章}
\newcommand{\secref}[1]{第\ref{#1}节}
\newcommand{\pref}[1]{第\pageref{#1}页}
\newcommand{\fref}[1]{图\ref{#1}}
\newcommand{\tref}[1]{表\ref{#1}}

% 编号列表宏包,并自定义了三个列表
\usepackage[inline]{enumitem}
	\setlist[enumerate]{label=\arabic* - ,font=\bfseries,itemsep=0pt}
	\setlist[itemize]{label=$\bullet$,font=\bfseries,leftmargin=\parindent}
	\setlist[description]{font=\bfseries\uline}

\newenvironment{fead}{\setlength{\parskip}{0pt}
	\begin{description}[font=\bfseries\uline,labelindent=\parindent]
		\setlength{\itemsep}{0pt}\setlength{\parsep}{0pt}\setlength{\parskip}{0pt}}
	{\end{description}}
% 带宽度的
\newenvironment{para}{\setlength{\parskip}{0pt}
	\begin{description}[font=\bfseries\ttfamily]
		\setlength{\itemsep}{0pt}\setlength{\parsep}{0pt}\setlength{\parskip}{0pt}}
	{\end{description}}
\newenvironment{feae}{\setlength{\parskip}{0pt}
	\begin{enumerate}[font=\bfseries,labelindent=0pt]}
	{\end{enumerate}}
\newenvironment{feai}{\setlength{\parskip}{0pt}
	\begin{itemize}[font=\bfseries]
		\setlength{\itemsep}{0pt}\setlength{\parsep}{0pt}\setlength{\parskip}{0pt}}
	{\end{itemize}}
\newenvironment{inlinee}
{\begin{enumerate*}[label=(\arabic*), font=\rmfamily, before=\unskip{:},itemjoin={{;}},itemjoin*={{,以及:}}]}
	{\end{enumerate*}。}

% 目录和章节样式
\usepackage{titlesec}
\usepackage{titletoc}   % 用于目录

\titlecontents{chapter}[1.5em]{}{\contentslabel{1.5em}}{\hspace*{-2em}}{\hfill\contentspage}
\titlecontents{section}[3.3em]{}{\contentslabel{1.8em}}
	{\hspace*{-2.3em}}{\titlerule*[8pt]{$\cdot$}\contentspage}
\titlecontents*{subsection}[2.5em]{\small}{\thecontentslabel{} }{}{, \thecontentspage}[;\qquad][.]
% 章节样式
\setcounter{secnumdepth}{3} % 一直到subsubsection
\newcommand{\chaformat}[1]{%
	\parbox[b]{.5\textwidth}{\hfill\bfseries #1}%
	\quad\rule[-12pt]{2pt}{70pt}\quad
	{\fontsize{60}{60}\selectfont\thechapter}}
\titleformat{\chapter}[block]{\hfill\LARGE\sffamily}{}{0pt}{\chaformat}[\vspace{2.5pc}\large
	\startcontents\printcontents{}{1}{\setcounter{tocdepth}{2}\songti}]
%\titleclass{\section}{top}
%\titleformat{\section}{\Large\bfseries}{\thesection}{0.5em}{}
\titleformat*{\section}{\centering\Large\bfseries}
\titleformat{\subsubsection}[hang]{\bfseries\large}{\rule{1.5ex}{1.5ex}}{0.5em}{}
% 扩展章节
\newcommand{\starsec}{\noindent\fbox{\S\textit{注意:本章节是一个扩展阅读章节。}}
	\\ \mbox{}}

\renewcommand{\contentsname}{目录}
	\renewcommand{\tablename}{表}
	\renewcommand\arraystretch{1.2}	% 表格行距
	\renewcommand{\figurename}{图}
% 设置不需要浮动体的表格和图像标题
\setlength{\abovecaptionskip}{5pt}
\setlength{\belowcaptionskip}{3pt}
\makeatletter
\newcommand\figcaption{\def\@captype{figure}\caption}
\newcommand\tabcaption{\def\@captype{table}\caption}
\makeatother
% 图表
\usepackage{array,multirow}
  \setlength\extrarowheight{2pt} % 行高增加
\usepackage{longtable}
\usepackage{graphicx}
  \graphicspath{{./tikz/}}
% 页面修正宏包
\usepackage{geometry}
\geometry{top = 1in}

% 代码环境
\usepackage{listings}
% Avoid copy line numbers of the listing code (Invalid for SumatraPDF Reader)
\usepackage{accsupp}
	\newcommand{\emptyaccsupp}[1]{\BeginAccSupp{ActualText={}}#1\EndAccSupp{}}
% Color
\usepackage{xcolor}
  \newcommand{\scol}[1]{\colorbox{#1}{\rule{0em}{1ex}}\,#1}
	\definecolor{commentcolor}{RGB}{85,139,78}
	\definecolor{numbercolor}{RGB}{166,206,168}
	\definecolor{stringcolor}{RGB}{206,145,108}
	\definecolor{keywordcolor}{RGB}{34,34,250}
	\definecolor{backcolor}{RGB}{220,220,220}
	\definecolor{packagecolor}{RGB}{0,128,0}
	\definecolor{envicolor}{RGB}{185,70,15}
% LaTeX Code Style
\lstset{language=[LaTeX]TeX,
		basicstyle=\small\ttfamily,
		commentstyle=\color{commentcolor},
		keywordstyle=\color{keywordcolor},
		stringstyle=\color{stringcolor},
		showstringspaces=false,
		% Package/Tikz-Lib Using
		classoffset=0,
		morekeywords={begin,end,usetikzlibrary},
		keywordstyle=\color{keywordcolor},
		classoffset=1,
		morekeywords={article,report,book,
			xeCJK,tikz,
			calc},
		keywordstyle=\color{packagecolor},
		classoffset=2,
		morekeywords={document,tikzpicture},
		keywordstyle=\color{envicolor},
		% Line Number Style
		numbers=left,
		numberstyle=\tiny\emptyaccsupp,
		stepnumber=1,
		% Frame and Background Color
		frame=single,
		framerule=0pt,
		backgroundcolor=\color{backcolor},
		% Spaces
		% belowskip=\medskipamount,
		emptylines=1,
		escapeinside=``}

\lstnewenvironment{latex}[1]{\lstset{#1}}{}
\newcommand{\latexline}[1]{{\lstinline[language=TeX,basicstyle=\small\ttfamily]{#1}}}

% Tikz Code
\lstdefinelanguage{tikzlang}{
	classoffset=0, % 蓝色的keyword
	morekeywords={begin,end,newcommand,
		draw,node,coordinate,tikzstyle,foreach},
	keywordstyle=\color{keywordcolor},
	classoffset=1, % 棕色的其他关键字
	morekeywords={tikzpicture,grid,at,
		thick,thin,very,ultra,
		red,green,yellow,blue,cyan,magenta,black,
		    gray,darkgray,lightgray,brown,lime,
		    olive,orange,pink,purple,teal,violet,white},
	keywordstyle=\color{envicolor},
	morecomment=[l]{\%},
	morecomment=[s]{/*}{*/},
	morestring=[b]',
	% Escape
	escapeinside=``
}
\lstnewenvironment{tikzcode}[1]{\lstset{language=tikzlang,basicstyle=\small\ttfamily,
		breaklines=true,%backgroundcolor=\color{white},
		linewidth=0.7\linewidth,#1}}{}

% 附录
\usepackage{appendix}

% 行号
\usepackage{lineno}

% 代码输入环境
\usepackage{verbatim,xcolor}
\newbox\savedlines
\newtoks\savedtokens
\makeatletter
\def\codeshow{%
\global\savedtokens={}%
\def\verbatim@processline{%
{\setbox0=\hbox{\the\verbatim@line}%
\hsize=\wd0
\the\verbatim@line\par}%
\global\savedtokens=\expandafter{\the\expandafter\savedtokens\the\verbatim@line^^J}}%
\@tempswatrue
\setbox0=\vbox\bgroup\parskip=0pt\topsep=0pt\partopsep=0pt
\verbatim}
\def\endcodeshow{\endverbatim%
\unskip\setbox0=\lastbox\egroup
\global\setbox\savedlines=\box0
\addvspace{1em}\par\noindent%
\colorbox{lightgray}{%
\begin{minipage}{.55\textwidth}{\usebox\savedlines}\end{minipage}}%
\hfill\fbox{\parbox{.40\textwidth}%
{\scantokens\expandafter{\the\savedtokens\unskip\endinput}}}%
\par\addvspace{1em}}
\makeatother

% 引用
\usepackage{hyperref}
\hypersetup{colorlinks, bookmarksopen = true, bookmarksnumbered = true}
